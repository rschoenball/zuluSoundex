\documentclass[12pt,a4paper]{article}
\usepackage[latin1]{inputenc}
\usepackage{amsmath}
\usepackage{amsfonts}
\usepackage{amssymb}
\usepackage{graphicx}
\author{Raphael Schönball}
\title{big Data and politics}
\begin{document}
	\begin{center}
	\begin{huge}
		AAABBB
	\end{huge}
	
	\begin{Large}
		Big data for more natural laws
	\end{Large}
	\end{center}
	
	\section{Introduction and task}
	
	For maintaining the functuality of democracies, it is crucial that politicians can communicate their decisions and actions. For a politician, it is simple to advocate rules which can be understood right away. However, this tendency leads to a simplified and non-differentiated legislation. For instance, it is easy to advocate one minimum wage for everyone rather than considering the financial margins of different economic sectors. In one sector, the law can lead to a more equal distribution of wealth among the employees, in other sectors, companies have to decrease their number of employees in order to survive. One minimum wage is too low for one company, but too high for another company. The data which are available today offer a whole new perspective on this issue. Why don't we use data in order to develop laws which consider each individual situation? Laws do not always have to determine non-differentiated and unnatural lines for everyone which tarnish their original meaning, they can adapt to the natural and individual reality they are applied to in order to fortify their meanings and to clarify their impacts from all unintended effects. \\
	Nowadays, many laws apply step-functions on everyone's reality: If your car fulfils the European emission standard x of 5 $(x=5)$, then you can enter more areas $(y=1)$ as opposed to when your car fulfils another European emission standard $x\leq4 \Rightarrow y=0$. Some laws apply linear functions to reality, e.g. the income tax ($y$) you pay is (when it is in a certain range) linearly dependend on your income ($x$)). Laws like these are easy to communicate, however they do not take into account that nature is more complex. Reality is most of the times not distributed according to step-functions or linear functions but according to the natural distribution or the fisher distribution or any other (in most cases continuous) distribution. This is not considered when lines are created in the legislation process and reminds one of straight border lines which were drawn by colonialistic powers on maps, separating peoples, cultures and religions in the most unnatural way. Applying rules to our society which fits better to its natural being can lead to a clearer meaning and a more efficient impact of the laws which society has agreed on and can hereby strengthen democracy.
	
	\section{Information retrieval and preparation}
	
	The study starts with the gathering of all information which are available about this topic so far. To this would also count a future visit of the DataforPolicy-conference (dataforpolicy.org). Then, possibilities are screened for practical applications of smart policies. These first ideas might lead to further work: \\
	Some cities in Germany are dealing with their polluted air. One solution is excluding all cars which do not fulfil a certain European emission standard. However, cars can perform differently than the standard they were assigned to. Data analysis connected with IoT features could help to exclude only the cars, which really pollute the air in cities. It is discussed nowadays whether one should introduce sensors in the exhaust pipes of cars for having more information on the cars' performance on the road. These values could be taken in order to determine whether a certain car is allowed in certain areas or not, according to how the car ranks in the distribution of cars in terms of pollution. This type of measurement is more close to reality than the existing classification procedure.
	 
	Another idea for a practical research project is finding the information of a company (like earnings, savings, number of employees) in order to assess of how the minimum-wage in a company could be calculated so that the company is able to invest and to grow but at the same time that the wage of the employees is not too low in comparison to the company's wealth in order to develop a more individual minimum wage. This minimum wage could also be a number of competition for companies in the advertising for new employees.
	
	There are especially many examples for unnatural policies in the field of taxation.
	
	Available data must be gathered and prepared for analysis.
	
	\section{Analysis and gathering of results}
	
	An existing law which implies e.g. a step-function is taken and compared to the underlying real distribution in society (for the European emissions standard, studys are assembled which explore the emission of cars on the road as opposed to in the testing atmosphere, these data could be taken for creating the distribution of emissions in reality). A model will be introduced of how to change the law in a smart law, based on easily measureable parameters. A new proposition of the text of the law is written. The effects of the changed law can be assessed and compared to the effects of the existing law.
	
	\section{Ethical discussion}
	
	In the resulting paper, an ethical debate must be included: Based on how exact and surveilling the measurements are which are the basis of a Smart Law, it must be discussed whether and how these measurements can lead to a society of perfect surveillance. The impacts of the existing example, that a German insurance company offers car insurances which fees for the consumer are based on his driving behavior which is measured by his or her cell phone data,  It must be discussed how much information state agencies are allowed to save about the individual citizens. Furthermore, potential misclassifications caused by data errors must be recognized and considered. One must think about the development period of a law just like the development period of an algorithm (can there be a law 2.0 ?). Furthermore, potential corruptions of the system must be screened. The safety of the data bases which could contain many information about citizens must be discussed as well as the threats arising from modified data collectors (e.g. the sensor in the exhaustion pipe can be modified etc.).
	\paragraph*{}
	After assembling the results in a convenient format (paper, poster, presentation etc.), the results are ready to be spread.
\end{document}
